% !TeX spellcheck = en_US
% This work is licensed under the Creative Commons
% Attribution-NonCommercial-ShareAlike 4.0 International License. To view a copy
% of this license, visit http://creativecommons.org/licenses/by-nc-sa/4.0/ or
% send a letter to Creative Commons, PO Box 1866, Mountain View, CA 94042, USA.

\section{2th Homework ALGZTH}
\subsection{H23}
Let $f \in K[X]$ irreducible und $[L:K]$ finite and coprime (teilerfremd?) to $\deg(f)$.\\
Claim: $f$ is irreducible in $L[X]$.

\begin{proof}
	\begin{itemize}
		\item $[L:K] = \dim_K L < \infty$ and $\gcd([L:K], \deg(f)) =1$. For a polynomial $f$ to be reducible means, that it will split into linear factors and these are somewhat like the field dimension, which is needed to do that. If the field dimension and the polynomial degree are coprime, we wouldnt find a ``suitable base'', which would make the polynomial split into linear factors in $L[X]$, and this would be the claim.
		%TODO make precise.
		Let $f \in K[X] \and \gcd([L:K], \deg(f)) = 1$ (field degree is finite!) this implies that $f$ cant be split into linear factors over $L$ aka irreducible over $L[X]$.
	\end{itemize}
\end{proof}

%%%%%%%%%%%%%%%%%%%% Aufgabe 2 %%%%%%%%%%%%%%%%%%%%%%%%%%%%%%%%%%%%%%%%%%%%%%%%%%
\subsection{H24}
Find the splitting field over $\Q$ and determine the degree: 
\begin{enumerate}
	\item $f = X^2 + X +1$
	\item $g = X^3 + X^2 + X + 1$
	\item $h = X^4 + X^3 + X^2 + 1$
\end{enumerate}

\begin{solution}
	\begin{enumerate}
		\item 
		\item With trying, found for $g$ the roots: $X_1 = \ii, X_2 = -\ii, X_3 = -1$, the splitting field should be $\Q(i)$. %TODO confirm!!!
		\item with the help of Mathematica 11.3, I found the roots %TODO find a method by hand?
		\begin{align*}
			X_i &= \zeta_5^i \text{ for } i = 1,2,3,4
		\end{align*}
		The splitting field for $h$ should be $\Q(\zeta_5)$. %TODO verify!!!
		Maybe to similar to H8? Or maybe I can use H8's method to calc it by hand?
	\end{enumerate}
\end{solution}

%%%%%%%%%%%%%%%%%%%% Aufgabe 3 %%%%%%%%%%%%%%%%%%%%%%%%%%%%%%%%%%%%%%%%%%%%%%%%%%
\subsection{H25}
Determine the splitting field degree of $f = X^4 + 1$ over $\Q$ and over $\Q(\sqrt{2})$. In V18 we showed already that $f$ ist irreduzibel over $\Q$.
\begin{solution}
	Let $\Split_i, i =1,2$ denote the splitting fields of $X^4 +1$ over $\Q$ and over $\Q(\sqrt{2})$.
	\begin{itemize}
		\item Then we can split into linear factors in $\Split_1$.
		So we have
		\begin{align*}
		X^ +1 &= (X^2 -\ii)(X^2 +\ii) = (X - \sqrt{\ii})(X+ \sqrt{\ii})(X- \sqrt{-\ii})(X- \sqrt{-\ii})\\
		&= (X-e^{\pi\ii /4})(X+e^{\pi\ii /4})(X-e^{3\pi\ii/4})(X+e^{\ii 3\pi\ii/4}).
		\end{align*}
		Because $\Split$ ist the splitting field of $X^4 +1$ over $\Q$ then $\Split_1$ has to be the smallest field containing $\Q$ and the roots of $X^4 +1$. So we get
		\[
		\Split_1 = \Q(e^{\pi\ii /4}, -e^{\pi\ii /4}, -e^{3\pi\ii/4}, e^{3\pi\ii/4}).
		\]
		Since $e^{3\pi\ii/4}$ can be generated by taking the cube root of $-e^{\pi\ii/4}$, we get $\Split_1 = \Q(e^{\pi\ii/4}) = \Q(\zeta_4)$. Since also $1 \in \Q$, we dont need to adjoin 1. And so we get $\Split_1 = \Q(\ii, \sqrt{2})$. The degree should be $[\Q(\ii, \sqrt{2}):\Q] = 4$
		\item The splitting field $\Split_2$ should be $\Q(i)$, because $\sqrt{2}$ is already part of the field and we do not need to adjoin it separate. Then the degree should be $[\Q(\ii, \sqrt{2}):\Q(\sqrt{2})] = 2$ %TODO verity!
	\end{itemize}
\end{solution}