% !TeX spellcheck = en_US
% This work is licensed under the Creative Commons
% Attribution-NonCommercial-ShareAlike 4.0 International License. To view a copy
% of this license, visit http://creativecommons.org/licenses/by-nc-sa/4.0/ or
% send a letter to Creative Commons, PO Box 1866, Mountain View, CA 94042, USA.

\section{2th Homework ALGZTH}
\subsection{H23}
Let $f \in K[X]$ irreducible and $[L:K]$ finite and coprime to $\deg(f)$.\\
Then is $f$ irreducible in $L[X]$.

\begin{proof}[Proof.]\ %TODO rename Beweis automatical to Proof. without adding per hand.
	Suppose $\alpha$ is a root of $f$ in an algebraic closure of $L$. Multiplication of the field degree gives
	\begin{align*}
		[L(\alpha):K] &= [L(\alpha): L] \mal [L:K]\\
		&= [L(\alpha):L(\alpha)]\mal [K(\alpha):K].
	\end{align*}
	It holds that
	\begin{align*}
		[L(\alpha):L] &= \frac{[L(\alpha):K(\alpha)]\mal [K(\alpha): K]}{[L:K]}\\
		&= \frac{[L(\alpha):K(\alpha)\deg(f)]}{[L:K]}.
	\end{align*}
	Because $[L:K] \and \deg(f)$ is coprime, $[L:K]$ is a divisor of $[L(\alpha):K(\alpha)]$. But on the other hand holds always that $[L(\alpha):K(\alpha)] \le [L:K]$, and here we have equality. It follows that $[L(\alpha):L] = \deg(f)$ and this gives us the claim, that $f$ is irreducible in $L[X]$.
\end{proof}

%%%%%%%%%%%%%%%%%%%% Aufgabe 2 %%%%%%%%%%%%%%%%%%%%%%%%%%%%%%%%%%%%%%%%%%%%%%%%%%
\subsection{H24}
Find the splitting field over $\Q$ and determine the degree: 
\begin{enumerate}
	\item $f = X^2 + X +1$
	\item $g = X^3 + X^2 + X + 1$
	\item $h = X^4 + X^3 + X^2 + 1$
\end{enumerate}

\begin{solution}\
		\begin{enumerate}
			\item We find the alternate form $f = (x+\frac{1}{2})^2 + \frac{3}{4}$ and from there we can conclude that the 	splitting field should be $\Q(\zeta_3)$. $f$ is also the 3rd cyclotomic polynomial. We defined cyclotomic polynomial in GEO 2.7.7, from that follows that $\zeta_3$. The field degree of $[\Q(\zeta_3):\Q] = 3$.
			\item We find the alternate form $g = (x+1)(x^2 +1) = \phi_4(x+1)$, so we get as splitting field $\Q(i)$, the 1 can be generated by $\ii$. The field degree should be $[\Q, \Q] = 2$.
			\item We find that $h = \phi_5$ the 5th cyclotomic polynomial (same argument as for $f$) and therefore $\Q(\zeta_5)$ is the splitting field. The field degree is $[\Q(\zeta_5), \Q] = 4$.
		\end{enumerate}
\end{solution}

%%%%%%%%%%%%%%%%%%%% Aufgabe 3 %%%%%%%%%%%%%%%%%%%%%%%%%%%%%%%%%%%%%%%%%%%%%%%%%%
\subsection{H25}
Determine the splitting field degree of $f = X^4 + 1$ over $\Q$ and over $\Q(\sqrt{2})$. In V18 we showed already that $f$ ist irreduzibel over $\Q$.
\begin{solution}\
	Let $\Split_i, i =1,2$ denote the splitting fields of $X^4 +1$ over $\Q$ and over $\Q(\sqrt{2})$.
	\begin{itemize}
		\item Then we can split into linear factors in $\Split_1$.
		So we have
		\begin{align*}
		X^ +1 &= (X^2 -\ii)(X^2 +\ii) = (X - \sqrt{\ii})(X+ \sqrt{\ii})(X- \sqrt{-\ii})(X- \sqrt{-\ii})\\
		&= (X-e^{\pi\ii /4})(X+e^{\pi\ii /4})(X-e^{3\pi\ii/4})(X+e^{\ii 3\pi\ii/4}).
		\end{align*}
		Because $\Split$ ist the splitting field of $X^4 +1$ over $\Q$ then $\Split_1$ has to be the smallest field containing $\Q$ and the roots of $X^4 +1$. So we get
		\[
		\Split_1 = \Q(e^{\pi\ii /4}, -e^{\pi\ii /4}, -e^{3\pi\ii/4}, e^{3\pi\ii/4}).
		\]
		Since $e^{3\pi\ii/4}$ can be generated by taking the cube root of $-e^{\pi\ii/4}$, we get $\Split_1 = \Q(e^{\pi\ii/4}) = \Q(\zeta_4)$. Since also $1 \in \Q$, we dont need to adjoin 1. And so we get $\Split_1 = \Q(\ii, \sqrt{2})$. The degree should be $[\Q(\ii, \sqrt{2}):\Q] = 4$
		\item The splitting field $\Split_2$ should be $\Q(i)$, because $\sqrt{2}$ is already part of the field and we do not need to adjoin it separate. Then the degree should be $[\Q(\ii, \sqrt{2}):\Q(\sqrt{2})] = 2$ %TODO verity!
	\end{itemize}
\end{solution}