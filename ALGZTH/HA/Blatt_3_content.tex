% !TeX spellcheck = en_US
% This work is licensed under the Creative Commons
% Attribution-NonCommercial-ShareAlike 4.0 International License. To view a copy
% of this license, visit http://creativecommons.org/licenses/by-nc-sa/4.0/ or
% send a letter to Creative Commons, PO Box 1866, Mountain View, CA 94042, USA.

\section{2th Homework ALGZTH}
Let $L\mid K$ be a field extension and $p = \chara(K) \ge 0$.
\subsection{H38}
Determine the splitting field degree of the polynomial $f = X^4 +2X^2 -2$ over $\Q$. Is $f$ separable over $\Q$?
\begin{solution}
	\begin{itemize}
		\item Find an alternate form $f = (X^2+1)^2+1$ where we can obtain with substitution the roots easily. The roots are
		\begin{align*}
			X_{1,2} = \pm \sqrt{1-\ii} \qquad X_{3,4} = \pm \sqrt{-1+\ii}.
		\end{align*}
		With that we find the splitting field is $\Split = \Q(\sqrt{-1+\ii}, \sqrt{1-\ii})$ and the splitting field degree is $[\Q(\sqrt{-1+\ii}, \sqrt{1-\ii}), \Q] = 4$.\\ 
		\item With definition 6.1 in the lecture we need to check if the roots are simple, which means if the simplicity $\mu(f,X_i) \with i = 1,2,3,4$ of the root is 1. 
	\end{itemize}
\end{solution}

%%%%%%%%%%%%%%%%%%%% Aufgabe 2 %%%%%%%%%%%%%%%%%%%%%%%%%%%%%%%%%%%%%%%%%%%%%%%%%%
\subsection{H39}
Let $L = K(X)$ a rational function field. Let $a = \frac{f}{g} \in L \setminus K \with f,g \in K[X]$ coprime. To show:
\begin{align*}
	[L:K(a)] = \max\set{\deg(f),\deg(g)}
\end{align*}
\begin{proof}
	
\end{proof}

%%%%%%%%%%%%%%%%%%%% Aufgabe 3 %%%%%%%%%%%%%%%%%%%%%%%%%%%%%%%%%%%%%%%%%%%%%%%%%%
\subsection{H40}
Let $p > 0, a \in K \and f = X^p - X + a \in K[X]$. To show
\begin{enumerate}
	\item $f(X) = f(X+1)$ \label{H40_1}
	\item $f$ is separable.
	\item Every Wurzelkörper (this does not exists in the literature) of $f$ is a splitting field of $f$.
	\item Has $f$ not a roots in $K$, then is $f$ irreducible. 
\end{enumerate}
\begin{proof}
	\begin{enumerate}
		\item use V1, freshmans dream
		\item Yes, because with \ref{H40_1} we get that also $a +1, \dots, a+ p-1$ are roots.
		\item The Wurzelkörper has only one root and the splitting field has only all roots.
		\item wikipedia!
	\end{enumerate}
\end{proof}