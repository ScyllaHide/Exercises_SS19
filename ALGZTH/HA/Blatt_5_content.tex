% !TeX spellcheck = en_US
% This work is licensed under the Creative Commons
% Attribution-NonCommercial-ShareAlike 4.0 International License. To view a copy
% of this license, visit http://creativecommons.org/licenses/by-nc-sa/4.0/ or
% send a letter to Creative Commons, PO Box 1866, Mountain View, CA 94042, USA.

\section{5th Homework ALGZTH}
Let $L\mid K$ be a field extension and $p = \chara(K) \ge 0$.
\subsection{H70}
\begin{proof}\
	\begin{itemize}
		\item $L_1 \mid K, L_2\mid K$ finite and normal with II.1.4 implies $L_1 L_2 \mid K$ us normal. Now $\bar{K} \mid K$ is algebraic, $L_1\mid K, L_2 \mid K$ is separable and I.7.9 implies $L_1 L_2 \mid K$ is separable. With these two facts we get $L_1 L_2 \mid K$ is galois (also because $L_1 L_2 \mid K$ algebraic and $L_1, L_2 \subseteq L_1 L_2$)
		\item \ul{To show:} $L_1\cap L_2 = K \implies \Gal(L_1 L_2 \mid K) \cong \Gal(L_1 \mid K) \times \Gal(L_2 \mid K)$. Consider
		\begin{align*}
			f: \begin{cases}
				\Gal(L_1 L_2 \mid K) &\to \Gal(L_1 \mid K) \times \Gal(L_2 \mid K)\\
				\sigma &\mapsto (\sigma_{\mid L_1}, \sigma_{\mid L_2})
			\end{cases}
		\end{align*}
		$L_1 L_2 \mid K, L_1 \mid K, L_2 \mid K$ are normal and this implies that $f$ is a group homomorphism.
		\begin{itemize}
			\item \ul{$f$ injectiv:}
			\begin{align*}
				\ker(f) &= \set{\sigma \in \Gal(L_1 L_2 \mid K) \mid f(\sigma) = (\id_{L_1} = \sigma_{\mid L_1}, \id_{L_2} = \sigma_{\mid L_2})}\\
				&=\set{\id_{L_1 L_2} = K(L_1 \cup L_2)}
			\end{align*}
			(Because $\sigma$ is unique defined through the $L_1, L_2$ values.)
			\item \ul{$f$ surjectiv:} Consider $(\sigma_1 , \sigma_2) \in \Gal(L_1 \mid K) \times \Gal(L_2 \mid K)$. With Ü68 and $L_1 \cap L_2 = K$ we can continue $\sigma_1$ to $\sigma'_1 \in \Gal(L_1 L_2 \mid K)$ (with $\sigma'_{1\mid L_2} = \id_{L_2}$) and analog for $\sigma_2$. It holds
			\begin{align*}
				(\sigma'_1 \circ \sigma'_2)_{\mid L_1} = \sigma'_{1\mid L_1} \circ \sigma'_{2\mid L_1} \and
				(\sigma'_1 \circ \sigma'_2)_{\mid L_2} = \sigma'_{1\mid L_2} \circ \sigma'_{2\mid L_2}.
			\end{align*}
			Hence $f(\sigma'_1 \circ \sigma'_2) = (\sigma_1, \sigma_2)$ and $f$ is surjectiv. 
		\end{itemize}
	Therefore $f$ is a group isomorphism.
	\end{itemize}
\end{proof}

%%%%%%%%%%%%%%%%%%%% Aufgabe 2 %%%%%%%%%%%%%%%%%%%%%%%%%%%%%%%%%%%%%%%%%%%%%%%%%%
\subsection{H71}
\begin{proof}\
	It holds
	\begin{align*}
		\Q(\ii, \zeta_3, \sqrt{2}) &= \Q(\ii, \sqrt{2}, - \frac{1+\sqrt{3}\ii}{2})\\
		&= \Q(\sqrt{2}, \ii, \sqrt{3}\ii)\\
		&= \Q(\sqrt{2}, \ii, \sqrt{3}) \equiv L. 
	\end{align*}
	\begin{itemize}
		\item $\Q$ is separable, because $\Q$ is perfect (``vollkommen'') and $\sqrt{2}, \sqrt{3} \and \ii$ are algebraic.
		\item $\Q$ is normal, because $L$ is the splitting field of
		\begin{align*}
			(X^2 -2)(X^2+1)(X^2-3) := f_{\sqrt{2}}f_{\ii}f_{\sqrt{3}}
		\end{align*}
		and $L$ is separable (all factors are irreducible over $\Q$) $\implies L \mid \Q$ galois. The polynomials $f_{\sqrt{2}},f_{\ii},f_{\sqrt{3}}$ are iredducible with EISENSTEIN, normed and have $\sqrt{2},\sqrt{3}, \ii$ as roots and are therefore minimal polynomials. The automorphisms of $L$, which fix $\Q$ map only $\Q$-conjugated to each other and with that we get $\sigma_i \in \Gal(L\mid \Q), i =1,\dots,8$. These are ($\Q$-conjugate are: $\sqrt{3}\colon \pm \sqrt{3}$, $\sqrt{2}\colon \pm \sqrt{2} \and \ii\colon \pm \ii$) and with $\sigma_i: L \to L \quad \forall i$:
		\begin{align*}
			\begin{tabular}{c|c|c|c}
			\hline
			$i \and \sigma$               & $\sigma(\sqrt{3})$             & $\sigma(\sqrt{2})$              & $\sigma(\ii)$\\ 
			\hline 
			1 & $\sqrt{3}$ & $\sqrt{2}$ & $\ii$ \\ %(\implies \sigma = \id_{L})$\\
			\hline
			2 & $\sqrt{3}$ & $\sqrt{2}$ & $-\ii$\\
			\hline
			3 & $\sqrt{3}$ & $-\sqrt{2}$ & $\ii$\\
			\hline
			4 & $\sqrt{3}$ & $-\sqrt{2}$ & $-\ii$\\
			\hline
			5 & $-\sqrt{3}$ & $\sqrt{2}$ & $\ii$\\
			\hline
			6 & $-\sqrt{3}$ & $\sqrt{2}$ & $-\ii$\\
			\hline
			7 & $-\sqrt{3}$ & $-\sqrt{2}$ & $\ii$ \\
			\hline
			8 & $-\sqrt{3}$ & $-\sqrt{2}$ & $-\ii$ \\     
			\end{tabular}
		\end{align*}
		In the table we can see, which $\Q$-conjugate get mapped to which elements. The automorphism is uniquely determined, because $L$ is ``spanned'' with these elments. Because $L\mid \Q$ is galois, did we find all elments in $\Gal(L\mid \Q)$, then $\# \Gal(L\mid \Q) = [L\colon \Q] = 8$:
		\begin{align*}
			[L:\Q] &= [\Q(\sqrt{2},\sqrt{3})(\ii):\Q(\sqrt{2},\sqrt{3})]\mal[\Q(\sqrt{2})(\sqrt{3}):\Q(\sqrt{2})]\mal[\Q(\sqrt{2}):\Q]\\
			\overset{(\star)}&{=} 2\mal 2\mal 2 = 8
		\end{align*}
		$(\star)$ see minimal polynomials above and all subset-relations are proper!
		\item Let check if $\alpha := \sqrt{3}+\sqrt{2}+\ii$ is a primitive element in the field extension $L\mid \Q$, $L=\Q(\alpha)$ and use V65. It holds
		\begin{align*} %TODO
			\begin{tabular}{c|c|c|c|c}
				i & 1 & 2 & 3 & 4\\
				\hline
				$\sigma_i(\alpha)$ & $\sqrt{2}+\sqrt{3}+\ii$ & $\sqrt{2}+\sqrt{3}-\ii$ & $\sqrt{2}+\sqrt{3}-\ii$ & $\sqrt{2}-\sqrt{3}-\ii$ \\
				\hline
				i & 5 & 6 & 7 & 8\\
				\hline
				$\sigma_i(\alpha)$ & $-\sqrt{2}+\sqrt{3}+\ii$ & $-\sqrt{2}+\sqrt{3}-\ii$ & $-\sqrt{2}-\sqrt{3}+\ii$ & $-\sqrt{2}-\sqrt{3}-\ii$ \\
			\end{tabular}
		\end{align*}
		Hence $\abs{\set{\sigma(\alpha) \mid \sigma \in \Gal(L\mid \Q)}} = 8 = [L:\Q]$ and therefore $L=\Q(\alpha)$.
	\end{itemize}
\end{proof}

%%%%%%%%%%%%%%%%%%%% Aufgabe 3 %%%%%%%%%%%%%%%%%%%%%%%%%%%%%%%%%%%%%%%%%%%%%%%%%%
\subsection{H72}
\begin{proof}\
	We can use the Galois-correspondence, because $L\mid \Q$ is finite galois. This means the number of ``Zwischenkörper'' equals the number of subgroups of $\Gal(L\mid \Q)$, which are because of $\Gal(L\mid \Q) \cong \Z / 4 \Z$ equal the number of subgroups of $\Z / 4\Z$. Because $\Z / 4\Z$ is cyclic, $\Z / 4\Z$ has to every divisor of the his order exact one subgroup. The divisors of $\abs{\Z / 4\Z} = 4$ are 1,2 and 4. (Excluding the trivial subgroups, we find \emph{one} nontrivial subgroup.) With the Galois-correspondence we get that there is exact one non-trivial ``Zwischenkörper'' and therefore $\Q \subsetneqq M \subsetneqq L$.
\end{proof}