%\RequirePackage{ifluatex}
\documentclass[
paper=A4,fontsize=12pt,
BCOR=15mm,DIV=22,
%twoside=true,
headinclude=true,footinclude=false,
parskip=full,
numbers=noendperiod,
ngerman,fleqn,             % fleqn = linksbündige Ausrichtung von Formeln
bibliography=totoc, % Bibliography appears in Table of Contents (without a number)
toc=listof, % List of Figures and List of Tables appear in Table of Contents
cleardoublepage=empty,      % Vakantseiten ohne Paginierung
version = last
]{scrartcl}

\usepackage[ngerman]{babel}
\usepackage{lmodern}

%\usepackage{lua-visual-debug}
\usepackage{scrhack}
\usepackage{marginnote}
\usepackage{scrpage2}
\pagestyle{scrheadings}
\clearscrheadfoot
\ihead{\headmark}\ohead{\pagemark}
\automark[subsection]{section}
\setheadsepline{0.5pt}
%\usepackage{fontspec}

\usepackage{calc}

\usepackage{graphicx,xcolor}
\usepackage{tabularx}
\usepackage{booktabs}
\usepackage{enumerate}
\usepackage{enumitem}
\usepackage{amsmath,amsfonts,amssymb,mathtools}
\usepackage{bbm}
\usepackage{xfrac}
\usepackage{siunitx}
\sisetup{
	locale = DE,
	round-precision=3,
	table-format=1.3e+1,
	round-mode=places,
%	table-figures-integer = 1,
%	table-figures-decimal = 6
}

\usepackage{pgfplots}
\pgfplotsset{compat=newest}

\usepackage[%
font={small},
labelfont={bf,sf},
format=hang, % try plain or hang
margin=5pt,
]{caption}

\usepackage[%
bookmarks, % Create bookmarks
bookmarksopen=true, % Unfold bookmatk tree in PDF viewer when document is opened
bookmarksopenlevel=1, % Level of unfolding
bookmarksnumbered=true, % Number bookmarks
hidelinks, % do not highlight hyperlinks -- looks ugly
% Ansicht beim Öffnen
pdfpagelabels=true, % See manual...
plainpages=false, % See manual...
hyperfootnotes=true, % Hyperlinks for footnotes
hyperindex=true, % Indexeinträage verweisen auf Text
]{hyperref}
\usepackage{cleveref}

\newenvironment{bottompar}{\par\vspace*{\fill}}{\clearpage}

\let\oldepsilon\epsilon
\let\epsilon\varepsilon

\newcommand{\transpose}[1]{%
	\ifmmode%
		{#1}^\textrm T%
	\else%
		$#1^\textrm T$%
	\fi%
}
%Shortcuts math symbols

\newcommand{\N}{\mathbb{N}}
\newcommand{\R}{\mathbb{R}}

\makeatletter
\newread\infile
\def\preparetable#1{
	\@bsphack
	\bgroup
	\openin\infile=#1
	\let\\=\relax
	\newcount\linecnt
	\gdef\usetable{}
	\endlinechar=-1
	\@whilesw\unless\ifeof\infile\fi{%
		\advance\linecnt by \@ne
		\read\infile to \line
		\if\relax\line\relax\else
			\xdef\usetable{\usetable \the\linecnt & \line  \\}
		\fi
	}
	 \egroup
	\@esphack
}
\makeatother

\newcommand{\konturplotsamples}{300}

\begin{document}

%\frontmatter

% Titelpageseite
\begin{titlepage}
	{
		\flushleft\includegraphics[width=0.33\textwidth]{TU_Logo_SW.pdf}\\[-2mm]
		\rule{\textwidth}{0.5pt}\\[-3.7mm]
		\rule{\textwidth}{0.5pt}
	}
	
	\vspace{\stretch{1.4}}
	\centering\large
	{\huge\bfseries documentation}
	
	\vspace{\stretch{0.1}}
	{der Modulbegleitenden Aufgabe P2\\
		des Moduls NUM
	}
	
	\vspace{\stretch{0.5}}
	vorgelegt von
	
	\vspace{\stretch{0.1}}
	{\Large Pascal Lehmann}
	
	\vspace{\stretch{1.0}}
	Institut für numerische Mathematik\\
	Fachrichtung Mathematik\\
	Fakultät Mathematik und Naturwissenschaften\\
	der Technischen Universität Dresden\\
	2019
\end{titlepage}


\pagenumbering{roman}
\setcounter{page}{1}

\tableofcontents
\clearpage

\pagenumbering{arabic}
\setcounter{page}{1}

\section{intial value problem and methods for solution}
\subsection{intial value problem}
For an initial value problem (IVP) we have
\begin{align}
\notag 
a,b\in \R\text{, } f:[a,b]\times\R^m\to\R^m \text{ und } u_0\in\R^m
\end{align}
Ziel dabei ist, eine auf $(a,b)$ stetig differenzierbare Funktion 
\begin{align}
\label{eq:AWP}
u:[a,b]\to\R^m \text{, die } u'(x)=f(x,u(x)) \text{ mit } u(a)=u_0
\end{align}
erfüllt, zu finden.
In this documentation we have to solve the logistic equation \eqref{eq:log} and another equation \eqref{eq:gl}.
\begin{align}
\label{eq:log}
& u'=\beta (1-u)u & \text{ mit } & x\in (0,10] \text{, für }u(0)=\alpha > 0 \text{, } \beta > 0 \\
\label{eq:gl}
&u'=10(\sin(x)-\cos(5x))& \text{ mit }& x\in (0,10] \text{, für } u(0)= 10 
\end{align}
\subsection{Numerical methods}

\subsubsection{Explicit and implicit Euler-method}

\subsubsection{Improved Euler-method}


\section{Solution of the intial value problem}
\subsection{analytical solution}
\subsubsection*{$u_1$}
	We can solve \cref{eq:log} with speration of the variables and with the initial value $u(0) = \alpha$ and we get 
	\begin{align*}
		\int_{\alpha}^{u} \frac{1}{u' (1-u')} \d u' = \beta \int_0^t \d t'.
	\end{align*} 
	For the solution we need to use partial fraction expansion and after integration we get
	\begin{align*}
		\ln \left( \frac{u}{\alpha} \cdot \frac{1-\alpha}{a-u}\right) = \beta t.
	\end{align*}
	and after a bit algebra fun we find the solution:
	\begin{align}
		u(t) = \frac{e^{\beta t}}{\frac{1-\alpha}{\alpha} + e^{\beta t}\label{eq:u_1:sol}
	\end{align}
\subsubsection*{$u_2$
	The second function we can quickly solve by simple integration with the initial value $u(0) = 10$. After integration
	\begin{align*}
		u(x) = \gamma - 10 \left( \cos(x) + \frac{1}{5} \sin(x)\right) \label{eq:u_2:sol}
	\end{align*}
\section{Numerical results}
\subsection{$u_1$}
\subsubsection*{explicit Euler method}
	
\subsubsection*{improved polygon method}
\subsubsection*{implicit euler method}
\subsection{$u_2$}
\subsubsection*{explicit Euler method}
\subsubsection*{improved polygon method}
\end{document}