% !TeX spellcheck = en_US
% This work is licensed under the Creative Commons
% Attribution-NonCommercial-ShareAlike 4.0 International License. To view a copy
% of this license, visit http://creativecommons.org/licenses/by-nc-sa/4.0/ or
% send a letter to Creative Commons, PO Box 1866, Mountain View, CA 94042, USA.

\section{7th Homework STOCH}
%%%%%%%%%%%%%%%%%%%% Aufgabe 1 %%%%%%%%%%%%%%%%%%%%%%%%%%%%%%%%%%%%%%%%%%%%%%%%%%
\subsection{}
\begin{proof}
	\begin{enumerate}
		\item \begin{align*}
			\Corr(aX + b,cY+d) &= \frac{\Cov(aX+b, cY+d)}{\sqrt{\Var(aX+b)\Var(cY+d)}}\\
			&= \frac{ab\Cov(X,Y)}{a^2\Var(X)c^2\Var(Y)}\\
			&= \frac{\cancel{ac}\Cov(X,Y)}{\cancel{ac}\sqrt{\Var(X)\Var(Y)}}\\
			&= \Corr(X,Y)
		\end{align*}
		Where we have used
		\begin{align*}
			\Var(aX+b) &= \E[(aX+b)^2]- \E[aX+b]^2\\
			&= \text{ follows from lecture prop 5.13a}\\
			&= a^2(\E[X^2] - \E[X]^2) = a^2\Var(X)
		\end{align*}
		and 
		\begin{align*}
			\Cov(aX+b,cY+d) &= \E[(aX+b)(cY+d)]-\E[(aX+b)]\E[(cY+d)]\\
			&= \E[acXY + bcY + adX + bd] \\
			&- (a\E[X]+b)(c\E[Y]+d)\\
			&= ac\E[XY]+\cancel{bc\E[Y]} + \cancel{ad\E[X]} + \cancel{bd} - ac\E[X]\E[Y]\\
			&- \cancel{bc \E[Y]} - \cancel{ad\E[X]} - \cancel{bd}\\
			&= ac(\E[XY] - \E[X]\E[Y])\\
			&= ac\Cov(X,Y)
		\end{align*}
		\item
		\begin{enumerate}[label=]
			\item \ul{$\Rightarrow:$} It holds $\P(Y = aX+b) = \P(Y-aX+b = 0) = \P(Z=0)$. Define a r.v. $Z:= Y - aX+b$. Now construct $a \and b$:
			\begin{align*}
				a:= \frac{\sqrt{\Var(Y)}}{\sqrt{\Var(X)}} \quad b = \E[Y] - a\E[X]
			\end{align*}
			with that holds
			\begin{align*}
				Z &= Y - \sqrt{\frac{\Var(Y)}{\Var(X)}} X + \E[Y] - \sqrt{\frac{\Var(Y)}{\Var(X)}}\E[X]\\
				&= \sqrt{\frac{\Var(Y)}{\Var(X)}}(X-\E[X]) + (Y - \E[Y])
				\intertext{and now for the expected value}
				\E[Z] &= -\sqrt{\frac{\Var(Y)}{\Var(X)}}(\E[X] - \E[\E[X]] + (\E[Y] - \E[\E[Y]]))\\
				&= \sqrt{\frac{\Var(Y)}{\Var(X)}} \mal 0 + 0 = 0
			\end{align*}
			Hence we get
			\begin{align*}
				\Var(Z) 
				&= \E[Z^2] \\
				&=\E\sqbrackets{\frac{\Var(Y)}{\Var(X)} (X-\E[X])^2 - 2 \sqrt{\frac{\Var(Y)}{\Var(X)}} (X- \E[X])(Y-\E[Y]) + (Y-\E[Y])^2}\\
				&= \frac{\Var(Y)}{\cancel{\Var(X)}}\cancel{\Var(X)} - 2\sqrt{\frac{\Var(Y)}{\Var(X)}} \Cov(X,Y) + \Var(Y)\\
				&= \Var(Y) - 2\sqrt{\frac{\Var(Y)}{\cancel{\Var(X)}}}\cancel{\sqrt{\Var(X)}}\sqrt{\Var(Y)} + \Var(Y)\\
				&= 0
			\end{align*}
			From there follows $\P(Z = \E[Z] = 1)$ so to say (sts) $1= \P(Z=0) = \P(Y-aX-b = 0) = \P(Y = aX + b)$.
			\item \ul{$\Leftarrow$:} Let $\P(Y = aX+b) = 1$ sts $\P(Y-aX+b = 0) = 1$.
		\end{enumerate}
	\end{enumerate}
\end{proof}

%%%%%%%%%%%%%%%%%%%% Aufgabe 2 %%%%%%%%%%%%%%%%%%%%%%%%%%%%%%%%%%%%%%%%%%%%%%%%%%
\subsection{}
\begin{proof}
	\begin{enumerate}
		\item from definition of the marginal densities follows
		\begin{align*}
			f_X(x) &= \int_{\R} f_{X,Y} (x,y)\d y\\
			&= \int_{\R} \indi_{\Delta}(x,y)\d y\\
			&= \indi_{[-1,1]}(x)\int_x^{1-\abs{x}} \d y\\
%			&= 1 - \abs{x}
			&= \int_{\R} \indi_{[-1,1]}(x)\indi_{[0,1-\abs{x}]}(y)\d y\\
			&= \indi_{[-1,1]}(x)\int_x^{1-\abs{x}}\d y\\
			&= \indi_{[-1,1]}(x)\mal (1-\abs{x})\\
			% % % % % % % % % % % % % % % % % % % % % % % % % %
			f_Y(y) &= \int_{\R} f_{X,Y}(x,y)\d x\\
			&= \int_{\R} \indi_{\Delta}(x,y)\d x\\
			&= \int_{\R}\indi_{[-1,1]}(x)\indi_{[0,1-\abs{x}]}(x)\d x\\
			&= \indi_{[0,1-\abs{x}]} \int_{-1}^1\d x\\
			&= 2\mal \indi_{[0,1-\abs{x}]}(y)
		\end{align*}
		\item The conditional densities are
		\begin{align*}
			f_{X\mid Y}(X,Y) &= \frac{f_{X,Y}(X,Y)}{f_Y(Y)}\\
			&= \frac{\indi_{\Delta}(x,y)}{2\mal \indi_{[0,1-\abs{x}]}(y)}\\
			&= \frac{\indi_{[0,1-\abs{x}]}(x)}{2}\\
			f_{Y\mid X}(X,Y) &= \frac{f_{X,Y}(X,Y)}{f_X(Y)}\\
			&= \frac{\indi_{\Delta}(x,y)}{\indi_{[-1,1]}\mal (1-\abs{x})}\\
			&= \frac{\indi_{[0,1-\abs{x}]}(y)}{1-\abs{x}}\\
		\end{align*}
		\item For the Covariance we have
		\begin{align*}
			\Cov(X,Y) &= \E[XY] -\E[X]\E[Y] = \E[XY] \text{ see below!}\\
			\E[X] &= \underbrace{\int_{-1}^1 x\mal (1-\abs{x})}_{=0}\indi_{[-1,1]}(x)\d x\\
			&= 0 \mal \indi ... = 0\\
			\E[Y] &= \int_0^{1-\abs{x}} y \mal 2 \dots \d y\\
			&= [y^2]_0^{1-\abs{x}} \mal \indi_{[0,1-\abs{x}]}\\
			&= (1-\abs{x})^2\indi_{[0,1-\abs{x}]}\\
			\E[XY] &= \int_{y=0}^{1-\abs{x}} \brackets{\int_{x=-1}^1 xy \mal \cancel{2\mal 1/2} \indi_{[0,1-\abs{x}]} (y) \mal \indi_{[-1,1]\times[0,1-\abs{x}]}(x,y) \d x} \d y\\
			&= \int_{y=0}^{1-\abs{x}} \brackets{\int_{x=-1}^1 y x \indi_{[-1,1]\times[0,1-\abs{x}]} (x,y)\d x}\d y\\
			&= \int_{y=0}^{1-\abs{x}} y \mal \indi_{[-1,1]\times[0,1-\abs{x}]} \d y\\
			&= \int_{y=0}^{1-\abs{x}} y \mal \indi ... \d y\\
			&= (1-\abs{x})/2 \mal \indi_{[-1,1]\times[0,1-\abs{x}]}(x,y)
		\end{align*}
		\item Suppose $X \and Y$ are independent, then holds
		\begin{align*}
			f_{XY} &= f_{X\mid Y} \mal f_{Y\mid X}\\
			&= \indi_{\Delta}(x,y) = \indi_{[-1,1]}(x) \mal \indi_{[0,1-\abs{x}]}(y)\\
			f_{X\mid Y} \mal f_{Y\mid X} &=\frac{1}{2} \indi_{[-1,1]}(x) \mal \indi_{[0,1-\abs{x}]}(y) \mal \frac{1}{1-\abs{x}}\\
			&= f_{XY} \mal \frac{1}{2}\mal \frac{1}{1-\abs{x}}\\
			&\neq f_{XY} 
		\end{align*}
		and thats a contradiction to the independence of $X \and Y$ we have assumed. Therefore $X \and Y$ are dependent.
	\end{enumerate}
\end{proof}
%%%%%%%%%%%%%%%%%%%% Aufgabe 3 %%%%%%%%%%%%%%%%%%%%%%%%%%%%%%%%%%%%%%%%%%%%%%%%%%
\subsection{}
\begin{proof}
	\begin{enumerate}
		\item Let
		\begin{align*}
			f_Y(y) &= \int_{\R} f(x,y) \d x\\
			&= \int_{\R} e^{-\frac{x}{y}}e^{-y}\d x\\
			&= e^{-y}\int_{\R} e^{-\frac{x}{y}} \d x\\
			&= y e^{-y}\indi_{(0,\infty)}(y)
		\end{align*}
%		and this will give us the marginal distribution for $Y$:
%		\[
%			\P(X\in A \mid Y=y) = \int_A f_Y(y)\d y = ...
%		\]
		\item Let
		\begin{align*}
			f_{X\mid Y}(x,y) &= \frac{f(x,y)}{f_Y(y)}\\
			&= \frac{e^{-\frac{x}{y}}\cancel{e^{-y}}}{y\mal\cancel{e^{-y}}}\\
			&= \frac{1}{y}e^{-\frac{x}{y}} \label{eq:7_3:b_1} \tag{$\times$}\\
			\P(X \le x \mid Y=y) &= \int_0^x f_{X\mid Y}(x,y) \\
			\overset{\eqref{eq:7_3:b_1}}&{=} \int_0^x 1/y e^{\frac{x}{y}} \mal \indi_{(0,\infty)\times(0,\infty)}(x,y)\d x\\
			&= y e^{-y}\indi_{(0,\infty)\times(0,\infty)}(x,y)
		\end{align*}
		\item Finally $\P(X>1 \mid Y=y)$:
		\begin{align*}
			\P(X>1\mid Y=y) &= \int_1^{\infty} f_{X\mid Y}(x,y)\d x\\
			&= \int_1^{\infty} \frac{1}{y} e^{-\frac{x}{y}}\d x\\
			&= \frac{1}{y} \int_1^{\infty} e^{-\frac{x}{y}} \d x\\
			&= e^{-\frac{1}{y}}
		\end{align*}
	\end{enumerate}
\end{proof}
%%%%%%%%%%%%%%%%%%%% Aufgabe 4 %%%%%%%%%%%%%%%%%%%%%%%%%%%%%%%%%%%%%%%%%%%%%%%%%%
\subsection{}
\begin{proof}
	\begin{enumerate}
		\item Let $X\sim \Poi(\lambda)$. To show: $\E[Xf(X)] = \lambda \E[f(X+1)]$.
		\begin{align*}
			\E[Xf(X)] &= \sum_{x=0}^{\infty}x\mal f(x)\frac{e^{-\lambda}\lambda^x}{x!}\\
			&= \sum_{x=1}^{\infty} f(x)\frac{e^{-\lambda}\lambda^x}{(x-1)!}\\
			&= \sum_{x=0}^{\infty}f(x+1)\frac{e^{-\lambda}\lambda\lambda^x}{x!}\\
			&= \lambda \mal \sum_{x=0}^{\infty} \frac{f(x+1)e^{-\lambda}\lambda^x}{x!}\\
			&= \lambda \sum_{x=0}^{\infty} f(x+1)\frac{e^{-\lambda}\lambda^x}{x!}\\
			&= \lambda\E[f(X+1)]
		\end{align*}
		\item TO show: $\E[Xf(X)] = \lambda \E[f(X+1)]$ $(\alpha)$ implies that $X\sim \Poi(\lambda) \with \lambda >0$.
		\begin{enumerate}
			\item \ul{LHS:} 
			\begin{align*}
				\E[Xf(X)] = \sum_{l=0}^{\infty} l \mal \indi_{\set{k}}(l) \rho(l) = k(\rho(k))
			\end{align*}
			\item \ul{RHS:}
			\begin{align*}
				\lambda\E[f(X+1)] = \lambda \P(X=k-1) = \lambda \rho(k-1)
			\end{align*}
			and this implies
			\begin{align*}
				\rho(k) = \frac{\lambda}{k}\rho(k-1)
			\end{align*}
			need to do induction over $k$ here and from that follows
			\begin{align*}
				\rho(k) = \frac{\lambda^k}{k!}\rho(0) \tag{$\times$}\label{eq:7_4:b}
			\end{align*}
			and the $e^{-\lambda}$ we get from the fact, that $\sum_{k=0}^{\infty} \rho(k) \overset{!}{=} 1$ and this implies that $\rho(0) = e^{-\lambda}$, where \eqref{eq:7_4:b} was used. Now set $f = \sum_{k=0}^{\infty} f(k)\indi_{\set{k}}$ and follows from $f_n = \sum_{k=0}^{\infty} f(k)\indi_{\set{k}} \xrightarrow{n \to \infty} f$. Set
			\begin{align*}
				f(X) &= \sum_{i=1}^{\infty} c_i \indi_{\set{i}}(x)\\
				f(X+1) &= \sum_{i=1}^{\infty} c_i \indi_{\set{i}}(x+1)
			\end{align*}
			And when we put this equation into $(\alpha)$, we get with BEPPO-LEVI for sums the claim.
		\end{enumerate}
	\end{enumerate}
\end{proof}

%%%%%%%%%%%%%%%%%%%% Aufgabe 5 %%%%%%%%%%%%%%%%%%%%%%%%%%%%%%%%%%%%%%%%%%%%%%%%%%
\subsection{}
\begin{proof}
	Consider $\P(X \in A \mid Y=y) = \int_A \frac{\rho_{X,Y}(x,y)}{\rho_Y(y)}$ and $\P(X \mid Y = y) = \int_{\R} \frac{\rho_{X,Y}(x,y)}{\rho_Y(y)}$ and then follows $f_{X\mid Y=y} = \frac{\rho_{X,Y}(x,y)}{\rho_Y(y)}$. Therefore
	\begin{align*}
		\E[X \mid Y=y] &= \int_{\R} x \frac{\rho_{X,Y}(x,y)}{\rho_Y(y)} \d x\\
		&= \int_{\R} x \frac{\rho_{X,Y}}{\rho_Y}\indi_{\rho_Y(y) >0} (x) \d x \\
		&+ \underbrace{\int_{\R} x \frac{\rho_{X,Y}}{\rho_Y} \int_{\R} x \frac{\rho_{X,Y}}{\rho_Y} \indi_{\rho_Y(y) =0} (x) \d x}_{=0, \text{ because null set}}
	\end{align*}
	and that is the claim.
\end{proof}

%\subsection{*}
%\begin{proof}
%	
%\end{proof}