% This work is licensed under the Creative Commons
% Attribution-NonCommercial-ShareAlike 4.0 International License. To view a copy
% of this license, visit http://creativecommons.org/licenses/by-nc-sa/4.0/ or
% send a letter to Creative Commons, PO Box 1866, Mountain View, CA 94042, USA.

\section{3rd Homework STOCH}
\subsection{}
Let $(\Omega, \F, \Meas)$ a probability space.
\begin{enumerate}
	\item Let $A,B \in F \with \Meas(A) >0$. Proof
	\begin{align*}
		\Meas(A\cap B \mid A\cup B) \le \P(A \cap B \mid A).
	\end{align*}
	\item Let $A_1 , A_2 , \dots , A_n \in \F$ independent events. Proof
	\begin{align*}
		\Meas(\set{\bigcap_{k=1}^n} A^C_k) \le \exp\brackets{- \sum_{k=1}^{n} \Meas(A_k)}
	\end{align*}
	\item Proof: Let $A,B,C \in F \with 0 < \P(C) < 1$. Then
	\begin{align*}
		\Meas(A\mid C) \ge \Meas(B \mid C) \and \Meas(A \mid C^C) \ge \Meas(B \mid C^C) \implies \Meas(A) \ge \Meas(B).
	\end{align*}
	\item Let $A, B \in \F$. Proof or disproof the claim
	\begin{align*}
		\Meas(A\cap B) = \Meas(A)\meas(B) \implies \Meas(B \mid A) = \Meas(A^C).
	\end{align*}
	\item Proof or disproof. There are events $A,B \in \F \with 0 < \meas(B) < 1, \Meas(A\mid B) = \Meas(A)$ and $\Meas(A \cap B) = \Meas(A \cup B)$.
	\begin{align*}
	\end{align*}
\end{enumerate}

\begin{proof}\
	\begin{enumerate}
		\item Start with definition:
		\begin{align*}
			\Meas(A \cap B \mid A \cup B) &= \frac{\Meas((A\cap B)\cap (A \cup B))}{\Meas(A \cup B)}\\
			&= \frac{\Meas((A\cap B)\cap A)}{\Meas(A \cup B)}\\
			&\le \frac{\Meas((A\cap B)\cap A)}{\Meas(A)}\\
			&= \Meas(A\cap B \mid A)
		\end{align*}
		\item With Lemma 3.16 is also the family of sets $A^C_1 , \dots, A^C_n \in \F$ stochastic independent and this implies $\Meas(\bigcap_{k=1}^n A_k^C) = \prod_{k=1}^{n} \Meas(A_k^C)$ ($\star$). Then we have
		\begin{align*}
			\Meas(\bigcap_{k=1}^n A^C_k) &\overset{\star}{=} \prod_{k=1}^n \Meas(A_k^C)\\
			&= \prod_{k=1}^n (1-\Meas(A_k))\\
			&\le \prod_{k=1}^n \exp(-\Meas(A_k)) \qquad \text{ use $e^{-x} \ge 1 - x$}\\
			&= \exp(-\sum_{k=1}^n \Meas(A_k))
		\end{align*}
		and that was the claim.
		\item Use again the definition of $\Meas(\star \mid \ast)$, we get
		\begin{align*}
			\Meas(A \mid C) = \frac{\Meas((A\cap C))}{\Meas(C)} &\ge \frac{\Meas((B\cap C))}{\Meas(C)} = \Meas(B\mid C)\\
			\Meas(A \mid C^C) = \frac{\Meas((A\cap C^C))}{\Meas(C^C)} &\ge \frac{\Meas((B\cap C^C))}{\Meas(C^C)} = \Meas(B\mid C^C)
		\end{align*}
		Putting both inequalities together, we get
		\begin{align*}
			\Meas(A \cap C) + \Meas(A \cap C^C) &\ge \Meas(B \cap C) + \Meas(B \cap C^C)\\
			\Meas((A \cap C) \cup (A \cap C^C)) &\ge \Meas((B \cap C) \cup (B \cap C^C))\\
			\Meas((A \cap C) \cup (A \setminus C)) &\ge \Meas((B \cap C) \cup (B \setminus C))\\
			\Meas(A) &\ge \Meas(B)
		\end{align*}
		\item %TODO
		\item %TODO
	\end{enumerate}
\end{proof}

%%%%%%%%%%%%%%%%%%%% Aufgabe 2 %%%%%%%%%%%%%%%%%%%%%%%%%%%%%%%%%%%%%%%%%%%%%%%%%%
\subsection{}
A conjurer has a fair and a double-headed coin in his pocket. He will throw the coin exact three times. He grabs in his pocket without looking and chooses a coin (random equal distributed). Nobody knows, which coin the conjurer grabs and tosses.
\begin{enumerate}
	\item The first toss shows head. How high is the probability that he tossed with the fair coin?
	\item The second toss is again head. How high is the probability that he tossed with the fair coin?
	\item The third toss is tail. How high is the probability that he tossed with the fair coin?
\end{enumerate}
\begin{proof}
	Let $F$ stand for fair, $U$ for unfair and $H, T$ for head and tail:
	\begin{align*}
		\Meas(F) = \Meas(U) = \frac{1}{2}\\
		\Meas(H \mid F) = \Meas(T \mid F) = \frac{1}{2}\\
		\Meas(H \mid U) = 1 \and \Meas(T \mid U) = 0
	\end{align*}
	\begin{enumerate}
		\item We observe
		\begin{align*}
			\Meas(F \mid H) &= \frac{\Meas(H \mid F)\Meas(F)}{\Meas(H)} = \frac{3}{4}\\
			\Meas(H) &= \Meas(H \mid F)\Meas(F) + \Meas(H \mid U)\Meas(U)\\
			&= \frac{1}{2} \cdot \frac{1}{2} + 1 \cdot \frac{1}{2}\\
			&= \frac{3}{4}
		\end{align*}
		\item ... %TODO
	\end{enumerate}
\end{proof}
%%%%%%%%%%%%%%%%%%%% Aufgabe 3 %%%%%%%%%%%%%%%%%%%%%%%%%%%%%%%%%%%%%%%%%%%%%%%%%%
\subsection{}
Let $(\Omega, \F, \Meas)$ and an index set $I \neq \emptyset$ and $A_i, i \in I$ independent events. Proof also that the family $\set{\Omega, A_i , i \in I}$ are independent.
\begin{solution}
	...
\end{solution}
%%%%%%%%%%%%%%%%%%%% Aufgabe 4 %%%%%%%%%%%%%%%%%%%%%%%%%%%%%%%%%%%%%%%%%%%%%%%%%%
\subsection{}(compare Corollary 3.18 lecture)
$(\Omega, \F, \Meas)$ a probability space and $\F_{i,j} \subseteq \F, 1 \le i \le n, 1 \le j \le m(i)$ independent, $\cap$-stable families with $\Omega \in \F_{i,j}$ for all $i,j$. Proof that the families
\begin{align*}
	\F_i^{\cap} := \set{F_{i,1} \cap \cdots \cap F_{i, m(i)} \colon F_{i,j} \in \F_{i,j}, 1 \le j \le m(i)} \quad 1 \le i \le n
\end{align*}
are $\cap$-stable, independent and also holds $\F_{i,1},\dots, \F_{i,m(i)} \subseteq \F_i^{\cap}$.

\begin{solution}
	...
\end{solution}
%%%%%%%%%%%%%%%%%%%% Aufgabe 5 %%%%%%%%%%%%%%%%%%%%%%%%%%%%%%%%%%%%%%%%%%%%%%%%%%
\subsection{*}
Find a counterexample, that independent families generated through $\sigma$-algebras are not independent.