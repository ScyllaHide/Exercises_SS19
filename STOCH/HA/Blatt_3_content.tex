% This work is licensed under the Creative Commons
% Attribution-NonCommercial-ShareAlike 4.0 International License. To view a copy
% of this license, visit http://creativecommons.org/licenses/by-nc-sa/4.0/ or
% send a letter to Creative Commons, PO Box 1866, Mountain View, CA 94042, USA.

\section{3rd Homework STOCH}
\subsection{}
Let $(\Omega, \F, \Meas)$ a probability space.
\begin{enumerate}
	\item Let $A,B \in F \with \Meas(A) >0$. Proof
	\begin{align*}
		\Meas(A\cap B \mid A\cup B) \le \P(A \cap B \mid A).
	\end{align*}
	\item Let $A_1 , A_2 , \dots , A_n \in \F$ independent events. Proof
	\begin{align*}
		\Meas(\set{\bigcap_{k=1}^n} A^C_k) \le \exp\brackets{- \sum_{k=1}^{n} \Meas(A_k)}
	\end{align*}
	\item Proof: Let $A,B,C \in F \with 0 < \P(C) < 1$. Then
	\begin{align*}
		\Meas(A\mid C) \ge \Meas(B \mid C) \and \Meas(A \mid C^C) \ge \Meas(B \mid C^C) \implies \Meas(A) \ge \Meas(B).
	\end{align*}
	\item Let $A, B \in \F$. Proof or disproof the claim
	\begin{align*}
		\Meas(A\cap B) = \Meas(A)\meas(B) \implies \Meas(B \mid A) = \Meas(A^C).
	\end{align*}
	\item Proof or disproof. There are events $A,B \in \F \with 0 < \meas(B) < 1, \Meas(A\mid B) = \Meas(A)$ and $\Meas(A \cap B) = \Meas(A \cup B)$.
	\begin{align*}
	\end{align*}
\end{enumerate}

\begin{proof}\
	\begin{enumerate}
		\item Start with definition:
		\begin{align*}
			\Meas(A \cap B \mid A \cup B) &= \frac{\Meas((A\cap B)\cap (A \cup B))}{\Meas(A \cup B)}\\
			&= \frac{\Meas((A\cap B)\cap A)}{\Meas(A \cup B)}\\
			&\le \frac{\Meas((A\cap B)\cap A)}{\Meas(A)}\\
			&= \Meas(A\cap B \mid A)
		\end{align*}
		\item With Lemma 3.16 is also the family of sets $A^C_1 , \dots, A^C_n \in \F$ stochastic independent and this implies $\Meas(\bigcap_{k=1}^n A_k^C) = \prod_{k=1}^{n} \Meas(A_k^C)$ ($\star$). Then we have
		\begin{align*}
			\Meas(\bigcap_{k=1}^n A^C_k) &\overset{\star}{=} \prod_{k=1}^n \Meas(A_k^C)\\
			&= \prod_{k=1}^n (1-\Meas(A_k))\\
			&\le \prod_{k=1}^n \exp(-\Meas(A_k)) \quad \text{ use $e^{-x} \ge 1 - x$}\\
			&= \exp(-\sum_{k=1}^n \Meas(A_k))
		\end{align*}
		and that was the claim.
		\item Use again the definition of $\Meas(\star \mid \ast)$, we get
		\begin{align*}
			\Meas(A \mid C) = \frac{\Meas((A\cap C))}{\Meas(C)} &\ge \frac{\Meas((B\cap C))}{\Meas(C)} = \Meas(B\mid C)\\
			\Meas(A \mid C^C) = \frac{\Meas((A\cap C^C))}{\Meas(C^C)} &\ge \frac{\Meas((B\cap C^C))}{\Meas(C^C)} = \Meas(B\mid C^C)
		\end{align*}
		Putting both inequalities together, we get (have used here that $0 <\Meas(C)<1$)
		\begin{align*}
			\Meas(A \cap C) + \Meas(A \cap C^C) &\ge \Meas(B \cap C) + \Meas(B \cap C^C)\\
			\Meas((A \cap C) \cup (A \cap C^C)) &\ge \Meas((B \cap C) \cup (B \cap C^C))\\
			\Meas((A \cap C) \cup (A \setminus C)) &\ge \Meas((B \cap C) \cup (B \setminus C))\\
			\Meas(A) &\ge \Meas(B)
		\end{align*}
		\item %TODO
		\item %TODO
	\end{enumerate}
\end{proof}

%%%%%%%%%%%%%%%%%%%% Aufgabe 2 %%%%%%%%%%%%%%%%%%%%%%%%%%%%%%%%%%%%%%%%%%%%%%%%%%
\subsection{}
A conjurer has a fair and a double-headed coin in his pocket. He will throw the coin exact three times. He grabs in his pocket without looking and chooses a coin (random equal distributed). Nobody knows, which coin the conjurer grabs and tosses.
\begin{enumerate}
	\item The first toss shows head. How high is the probability that he tossed with the fair coin?
	\item The second toss is again head. How high is the probability that he tossed with the fair coin?
	\item The third toss is tail. How high is the probability that he tossed with the fair coin?
\end{enumerate}
\begin{solution}
	Let $F$ stand for fair, $U$ for unfair and $H, T$ for head and tail:
	\begin{align*}
		\Meas(F) = \Meas(U) = \frac{1}{2}\\
		\Meas(H \mid F) = \Meas(T \mid F) = \frac{1}{2}\\
		\Meas(H \mid U) = 1 \and \Meas(T \mid U) = 0
	\end{align*}
	\begin{enumerate}
		\item We observe
		\begin{align*}
			\Meas(F \mid H) &= \frac{\Meas(H \mid F)\Meas(F)}{\Meas(H)} = \frac{1}{3}\\
			\Meas(H) &= \Meas(H \mid F)\Meas(F) + \Meas(H \mid U)\Meas(U)\\
			&= \frac{1}{2} \cdot \frac{1}{2} + 1 \cdot \frac{1}{2}\\
			&= \frac{3}{4}
		\end{align*}
		\item We observe
		\begin{align*}
			\Meas(F\mid H) &= \frac{\Meas(H \mid F)\Meas(F)}{\Meas(H\mid F)\Meas(F) + \Meas(H \mid F)\Meas(F)}\\
			&= \frac{\frac{1}{4}\frac{1}{2}}{\frac{1}{4}\mal \frac{1}{2} + 1 \mal 1 \mal \frac{1}{2}}\\
			&= \frac{1}{5} \approx 20\%
		\end{align*}
		\item We observe
		\begin{align*}
			\Meas(H \mid F) = \frac{1}{2}\frac{1}{2}\frac{1}{2}
		\end{align*}
		of course for 3 times head tossed with the fair coin. With that follows
		\begin{align*}
			\Meas(F \mid H) = \frac{\Meas(H \mid F)\Meas(F)}{\Meas(H \mid F)\Meas(F) + \Meas(H \mid )} %TODO
		\end{align*}
	\end{enumerate}
\end{solution}
%%%%%%%%%%%%%%%%%%%% Aufgabe 3 %%%%%%%%%%%%%%%%%%%%%%%%%%%%%%%%%%%%%%%%%%%%%%%%%%
\subsection{}
Let $(\Omega, \F, \Meas)$ and an index set $I \neq \emptyset$ and $A_i, i \in I$ independent events. Proof also that the family $\set{\Omega, A_i , i \in I}$ are independent.
\begin{proof}
	Independence is only defined for finite index sets, so we have to set $\emptyset \neq J \subset I, \with \abs{J} < \infty$. 
	We get
	\begin{align*}
		\Meas(\bigcap_{i \in J}A_i) = \prod_{i \in J}\Meas(A_i). \label{prob3}\tag{$\alpha$}
	\end{align*}
	Because all $A_i \in \F$ holds $A_i \subseteq \Omega$ and this implies $\bigcap_{i \in J} A_i = \bigcap_{i \in J} A_i \cap \Omega$. Now we know that $\Meas(\Omega) = 1$, so we can multiply both sides of equation \ref{prob3}
	\begin{align*}
		\Meas(\bigcap_{i \in J}A_i \cap \Omega) = \prod_{i \in J}\Meas(A_i)\Meas(\Omega)
	\end{align*}
	for every arbitrary subset $\emptyset \neq J \subset I \with \abs{J} < \infty$. 
	This shows that also the family $\set{\Omega, A_i , i \in I}$ is independent.
\end{proof}
%%%%%%%%%%%%%%%%%%%% Aufgabe 4 %%%%%%%%%%%%%%%%%%%%%%%%%%%%%%%%%%%%%%%%%%%%%%%%%%
\subsection{}(compare Corollary 3.18 lecture)
$(\Omega, \F, \Meas)$ a probability space and $\F_{i,j} \subseteq \F, 1 \le i \le n, 1 \le j \le m(i)$ independent, $\cap$-stable families with $\Omega \in \F_{i,j}$ for all $i,j$. Proof that the families
\begin{align*}
	\F_i^{\cap} := \set{F_{i,1} \cap \cdots \cap F_{i, m(i)} \colon F_{i,j} \in \F_{i,j}, 1 \le j \le m(i)} \quad 1 \le i \le n
\end{align*}
are $\cap$-stable, independent and also holds $\F_{i,1},\dots, \F_{i,m(i)} \subseteq \F_i^{\cap}$.

\begin{solution}
	...
\end{solution}
%%%%%%%%%%%%%%%%%%%% Aufgabe 5 %%%%%%%%%%%%%%%%%%%%%%%%%%%%%%%%%%%%%%%%%%%%%%%%%%
\subsection{*}
Find a counterexample, that independent families generated through $\sigma$-algebras are not independent.

\begin{proof}
	Consider the experiment: roll a dice with a six-sided fair dice. Set: $\Omega = \set{1,2,3,4,5,6}, \F = \pows(\Omega)$ and the partition into \\
	$\T = \set{Q_1, Q_2} \and \S = \set{S_1}$ with
	\begin{enumerate}[label=]
		\item $Q_1 := \set{1,4,2}$
		\item $Q_2 := \set{1,4,6}$
		\item $S_1 := \set{1,3}$
	\end{enumerate}
	First have to show $\T \and \S$ are independent. With definition $\T \and \S$ are independent, if
	\begin{align*}
		\Meas(Q_i \cap S_i) = \Meas(A_i)\Meas(S_i)\\
		\text{ for all } A_i \in \A, S_i \in \S \quad \forall i \in I \and \abs{I} < \infty.
		\intertext{Then}
		\Meas(Q_1 \cap S_1) = \Meas(\set{1}) = \frac{1}{6} = \frac{3}{6} \mal \frac{2}{6} = \Meas(Q_1)\Meas(S_1)\\
		\Meas(Q_2 \cap S_1) = \Meas(\set{1}) = \frac{1}{6} = \frac{3}{6} \mal \frac{2}{6} = \Meas(Q_2)\Meas(S_1)		
	\end{align*}
	this shows that $\T, \S$ are stochastic independent. Now let us check the generators $\sigma(\T) \and \sigma(\S)$. Lets assume $\sigma(\T) \and \sigma(\S)$ are stochastic independent, then we would need
	\begin{align*}
		\Meas(Q_i \cap S_i) = \Meas(Q_i)\Meas(S_i) \quad \forall A_i \in \sigma(\T), B_i \in \sigma(\S), i \in I \and \abs{I} < \infty.
	\end{align*}
	As we showed last term in MINT, intersections of elements in $\sigma$-algebra are again in the $\sigma$-algebra. Now we search for a counterexample with $Q_1 \cap Q_2 \in \sigma(\T)$. Consider
	\begin{align*}
		\Meas((Q_1 \cap Q_2)\cap S_1) 
		&= \Meas(\set{1,4} \cap \set{1,3}) = \Meas(\set{1}) = \frac{1}{6}\\
		\Meas(Q_1 \cap Q_2) &= \Meas(\set{1,4}) = \frac{2}{6}\\
		\Meas(S_1) &= \Meas(\set{1,3}) = \frac{2}{6}.
	\end{align*}
	Because $\frac{1}{6} \neq \frac{1}{9} = \frac{2}{6}\mal \frac{2}{6}$ $(\Meas((Q_1 \cap Q_2)\cap S_1) \neq \Meas(Q_1 \cap Q_2)\cap \Meas(S_1))$ and this gives that the $\sigma$-algebra generators $\sigma(\T) \and \sigma(\S)$ are stochastic dependent.
\end{proof}