% !TeX spellcheck = en_US
% This work is licensed under the Creative Commons
% Attribution-NonCommercial-ShareAlike 4.0 International License. To view a copy
% of this license, visit http://creativecommons.org/licenses/by-nc-sa/4.0/ or
% send a letter to Creative Commons, PO Box 1866, Mountain View, CA 94042, USA.

\section{10th Homework STOCH}
%%%%%%%%%%%%%%%%%%%% Aufgabe 1 %%%%%%%%%%%%%%%%%%%%%%%%%%%%%%%%%%%%%%%%%%%%%%%%%%
\subsection{}
\begin{proof}
	\begin{enumerate}
		\item TODO
		\item Follows quickly from the characteristic function definition (define $U= (u,0,\dots,0)$)
		\begin{align*}
			\varphi_X(U) &= \E[e^{\ii \scaProd{U}{X}}]\\
			&= \E[e^{\ii uX_1 + 0X_2 + \cdots + 0X_n}] \quad \text{definition scalar product}\\
			&= \E[e^{\ii uX_1}] = \varphi_{X_1}(u)
		\end{align*}
		\item Follows also quickly from the definition. We set $U = (u,u, \dots, u)$
		\begin{align*}
			\varphi_X(U) &= \E[e^{\ii \scaProd{U}{X}}]\\
			&= \E[e^{\ii u (X_1 + \cdots + X_n)}]\\
			&= \varphi_{X_1+\cdots+X_n}(u)
		\end{align*}
		\item We can solve this quick with the Residue Theorem (see ANAF for the physicsts from 2014 for example). We have
		\begin{align*}
		\frac{1}{\pi}\int_{-\infty}^{\infty} \frac{e^{\ii t x}}{x^2+1} \d x
		\end{align*}
		we can evaluate it by expressing the integral as a limit of contour integrals. So we can set with $t>0$ and define a contour $\mathscr{C}$ which goes along the real line from $-a$ to $a$ and counter clockweise along the semicircle centered at 0 from $a$ to $-a$ (need to go sure that the circle is bigger than 1, because we need to enclose $\ii$, so $a > 1$)
		\begin{align*}
		\frac{1}{\pi}\int_{\mathscr{C}} f(z) \d z = \int_{\mathscr{C}} \frac{e^{\ii t z}}{z^2+1}
		\end{align*}
		$e^{\ii t z}$ (because it is holomorphic at all finite points over the whole $\C$ plane) has no singularities only $z^2+1$ has in the denominator, so we find two singularities $z=\ii$ and $z = - \ii$. So only one of the two points is bounded by this contour and we have
		\begin{align*}
		\frac{e^{\ii t z}}{z^2+1} = \frac{e^{\ii t z}}{2 \ii (z- \ii)} - \frac{e^{\ii t z}}{2 \ii (z+ \ii)}
		\end{align*}
		So the residue of $f(z)$ at $z = i$ is
		\begin{align*}
		Res_{z=i} f(z) = \frac{e^{-t}}{2 \ii}
		\end{align*}
		We know that the contour splits up in a straight part and an arc part. The straight part we have already calculated. 
		\begin{align*}
			\int_{\mathscr{C}} f(z) \d z = 2 \pi \ii Res_{z=\ii} f(z) = \pi e^{-t}
			\intertext{we can see}
			\int_{\text{straight}} f(z) \d z + \int_{\text{arc}} f(z) \d z = \pi e^{-t}\\
			\int_{-a}^a f(z) = \pi e^{-t} - \int_{\text{arc}} f(z) \d z
		\end{align*}
		We need to use a few estimations and we have
		\begin{align*}
			\abs{\int_{\text{arc}} \frac{e^{\ii t z}}{z^2 +1} \d z} \le \pi a \mal \sup_{\text{arc}} \abs{\frac{e^{\ii t z}}{z^2+1}} \le \pi a \sup_{\text{arc}} \frac{1}{z^2+1}\le \frac{\pi a }{a^2 -1}
		\end{align*}
		and now
		\begin{align*}
			\lim_{a \to \infty} \frac{\pi a}{a^2 -1} = 0
		\end{align*}
		Where we have used that
		\begin{align*}
			\abs{e^{\ii t z}} = \abs{e^{\ii t \abs{z}(\cos(\phi) +\ii \sin(\phi))}} = \abs{e^{-t \abs{z} \sin(\phi) + \ii t \abs{z} \cos(\phi)}} = e^{-t \abs{z} \sin(\phi)} \le 1
		\end{align*}
		So we get
		\begin{align*}
			\int_{-\infty}^{\infty} \frac{e^{\ii t z}}{z^2+1} \d z = \pi e^{-t}.
		\end{align*}
		If we choose $t < 0$ we can handle things analog, but the arc simply winds around $-\ii$ and we get
		\begin{align*}
			\int_{-\infty}^{\infty} \frac{e^{\ii t z}}{z^2+1} \d z = \pi e^t
		\end{align*}
		Which gives us the end result for the CAUCHY-distribution
		\begin{align*}
			\varphi_X(u) = \frac{1}{\pi}\int_{-\infty}^{\infty} \frac{e^{\ii u z}}{z^2+1} \d z = e^{-\abs{u}}
		\end{align*}
		For convenience we left out the $\frac{1}{\pi}$ scalar and added it back in the last line.
	\end{enumerate}	
\end{proof}

%%%%%%%%%%%%%%%%%%%% Aufgabe 2 %%%%%%%%%%%%%%%%%%%%%%%%%%%%%%%%%%%%%%%%%%%%%%%%%%
\subsection{}
\begin{proof}
	TODO
\end{proof}
%%%%%%%%%%%%%%%%%%%% Aufgabe 3 %%%%%%%%%%%%%%%%%%%%%%%%%%%%%%%%%%%%%%%%%%%%%%%%%%
\subsection{}
\begin{proof}
	TODO
\end{proof}
%%%%%%%%%%%%%%%%%%%% Aufgabe 4 %%%%%%%%%%%%%%%%%%%%%%%%%%%%%%%%%%%%%%%%%%%%%%%%%%
\subsection{}
\begin{proof}
	\begin{enumerate}
		\item TODO 
		\item For the counterexample, let $X$ be a CAUCHY r.v. and $X=Y$ a.s.. The characterstic function (see 10.1d) was given by ($x_0$ is a displacement and $\alpha$ is scale factor) %TODO maybe i dont need the x_0 here and the scale alpha, but it makes it more general ...
		\begin{align*}
			\varphi_X (u) &= e^{\ii x_0 u + \alpha\abs{u}}
			\intertext{thus}
			\varphi_X (u) \varphi(u) &= \brackets{e^{\ii x_0 u + \alpha \abs{u}}}^2 = e^{2\ii x_0 u + 2\alpha\abs{u}}\\
			&= \E[e^{\ii u(2X)}] = \E[e^{\ii u (X+Y)}]\\
			&= \varphi_{X+Y}(u)
		\end{align*}
		and we know from previous constructed counterexamples that $X \and Y$ are \emph{not} independent.
	\end{enumerate}
\end{proof}