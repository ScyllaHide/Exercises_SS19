% This work is licensed under the Creative Commons
% Attribution-NonCommercial-ShareAlike 4.0 International License. To view a copy
% of this license, visit http://creativecommons.org/licenses/by-nc-sa/4.0/ or
% send a letter to Creative Commons, PO Box 1866, Mountain View, CA 94042, USA.

\section{2th Homework STOCH}
\subsection{}

\begin{solution}\
	\begin{enumerate}[label=\alph*)]
		\item \begin{enumerate}[label=\arabic*)]
			\item $r \le n +1$, because the drawing ends, when a ball is drawn for the second time and there are $n$ balls.
			\item $\Meas(x = r)$:
			\begin{align*}
				\begin{matrix}
				r = 3 & \Meas(x = 3) = \frac{n}{n} \mal \frac{n-1}{n}\mal \frac{2}{n}\\
				r = 4 & \Meas(x = 4) = \frac{n}{n} \mal \frac{n-1}{n}\mal \frac{n-2}{n} \mal \frac{3}{n}\\
				r & \Meas(x = r) = \frac{n}{n} \mal \frac{n-1}{n}\mal \cdots \frac{n-r+2}{n} \mal \frac{r-1}{n} = \frac{n\mal (n-1)\mal \cdots \mal (n-r+2)(r-1)}{n^r}
				\end{matrix}
			\end{align*}
			\item 
			\begin{align*}
				\Meas(x>r) &= 1 - (\Meas(x=r) +\Meas(x < r))\\
				\Meas(x<r) &= \Meas(x = (r-1)) + \Meas(x = (r-2)) + \cdots + \Meas(x=2)\\
				&= \sum_{i=2}^{r-1} \Meas(x=1)\\
				\Meas(x>r) &= 1 - (\Meas(x=r) + \sum_{i=2}^{r-1} \Meas(x=1)\\
				&= 1 - \sum_{i=2}^r \Meas(x=i)\\
				&= 1 - \sum_{i=2}^r \frac{n(n-1)\cdots(n-i +2)(i-1)}{n^i}
			\end{align*}
			The down counting ends with $x=2$, because for $x=1$ we draw a ball and the drawing ends and there is not a second drawing.
		\end{enumerate}
	\item Let $n$ be the different combinations for the 6 balls out of 49 balls: $\binom{49}{6}$. Set this in $\Meas(x=r), \for r= 3016$:
	\begin{align*}
		\Meas(x = 3016) &= \frac{\binom{49}{6}(\binom{49}{6}-1)\cdots \brackets{\binom{49}{6} - 3016 + 2}(3016 -1)}{\binom{49}{6}^{3016}}\\
		&= \frac{\binom{49}{6}(13989315)\cdots \brackets{13989815 - 3016 + 2}(3016)}{3016}
	\end{align*}
	\ul{Answer:} As one can see from the calculation above the nominator is significant bigger as the denominator and the probability will be ``really really small''. So the media was right!
	\end{enumerate}
\end{solution}

%%%%%%%%%%%%%%%%%%%% Aufgabe 2 %%%%%%%%%%%%%%%%%%%%%%%%%%%%%%%%%%%%%%%%%%%%%%%%%%
\subsection{}

\begin{proof}
	Let $\vec{k} \in \Omega$ fixed ($\Omega = \set{\vec{k} = (k_a)_{a \in E} \in \N_0^{\abs{E}}\colon \sum_{a \in E} k_a = n}$ - like we had defined in the lecture.). In the limit $N_a \to \infty$ holds
	\begin{align*}
		\binom{N_a}{k_a} &= \frac{N_a^{k_a}}{K_a !} \frac{N_a(N_a -1)\dots (N_a - k_a +1)}{N_a^{k_a}}\\
		&= \frac{N_a^{k_a}}{k_a !} 1\brackets{1 - \frac{1}{N_a}}\brackets{1 - \frac{2}{N_a}}\cdots \brackets{1 - \frac{k_a -1}{N_a}} \overset{N_a \to \infty}{\sim} \frac{N_a^{k_a}}{k_a !}.
	\end{align*}
	Here was the notation $``a(l) \sim b(l)'' \for l \to \infty$ for asymptotic equivalence used, this means for the statement $``\frac{a(l)}{b(l)}'' \to 1 \text{ in the limit } l \to \infty$. Then we have
	\begin{align*}
		\Hyper(n,N,\vec{k}) 
		&= \frac{\brackets{\prod_{a \in E} 
		\binom{N_a}{k_a}}}{\binom{N}{n}}
		\sim \frac{
				   \brackets{
                             \prod_{a \in E} \frac{N_a^{k_a}}
                             {k_a !}
                         }}{\frac{N^n}{n!}}\\
		&= \binom{n}{\vec{k}} \prod_{a \in E} \brackets{\frac{N_a}{N}}^{k_a} \to \Multi(n,\rho(a))(\vec{k}).
	\end{align*} 
	and this is the desired statement we wanted to proof.
\end{proof}
%%%%%%%%%%%%%%%%%%%% Aufgabe 3 %%%%%%%%%%%%%%%%%%%%%%%%%%%%%%%%%%%%%%%%%%%%%%%%%%
\subsection{}

\begin{solution}\ $n$ stand for the number of drawings and $N = b+w$, the number of ``balls'' in the urn.
	\begin{enumerate}[label=\alph*)]
		\item The event space is $\Omega_1 = \set{1,\dots, N}$ for the unsorted drawings.
		\item Can use the Uniform-distribution for probability measure $\Meas = \Uni(\Omega_1) = \frac{1}{\abs{\Omega_1}}$.
		\item The event space for this sorted case should be $\Omega_2 = \set{0,1}^n$, where $n$ is for the times of drawing and $n \in \N$.
		\item Set $X_1$ as random variable, defined as 
		\begin{align*}
			X_1: \Omega_1 \to \Omega_2 \with 
			X(x) = \begin{cases}
				1 \quad &x \text{ is black}\\
				0 \quad &x \text{ is white}
			\end{cases}
		\end{align*}
		with probability measures for black and white balls:
		\begin{align*}
			\Meas(X_1^{-1}(\set{0})) = \frac{\abs{w}}{\abs{\Omega_2}} = \frac{\abs{w}}{2^n}\\
			\Meas(X_1^{-1}(\set{1})) = \frac{\abs{b}}{\abs{\Omega_2}} = \frac{\abs{b}}{2^n}.
		\end{align*}
		\item Event space for colored case is $\Omega_3 = \set{0,\dots, N}$.
		\item The random variable for this case should be $X_2 : \Omega_2 \to \Omega_3$
		\begin{align*}
			X_2: \set{0,1}^n \to \set{0,1,..,N} \with
			X_2(t) = \begin{cases}
			0 \quad &\text{ in case 0 black ones drawn}\\
			\vdots &\vdots\\
			N \quad &\text{ in case $N$ black ones drawn}.
			\end{cases}
		\end{align*}
		Then we get the following probability measure:
		\begin{align*}
			\Meas(X = k) = \frac{\binom{b}{k}\binom{N-b}{n-k}}{\binom{N}{n}} = \frac{\binom{b}{k}\binom{w}{n-k}}{\binom{N}{n}}\\ \for k \in \Omega_3.
		\end{align*}
	\end{enumerate}
\end{solution}
%%%%%%%%%%%%%%%%%%%% Aufgabe 4 %%%%%%%%%%%%%%%%%%%%%%%%%%%%%%%%%%%%%%%%%%%%%%%%%%

\subsection{}

\begin{solution}\
	Again $b$ for black and $w$ for white balls.
	\begin{enumerate}[label=\alph*)]
		\item Which probability wins black? Black can win in two ways:
		\begin{enumerate}[label=\arabic*)]
			\item $B$ (black) wins instant (with the first try drawing a black ball), or \label{instant:winning}
			\item $B$ draws white, $W$ draws black ... then, the probability for Black winning is \label{longer:winning}
			\begin{align*}
			\Meas(B) &= \underbrace{\frac{b}{b+w}}_{\ref{instant:winning}} + \underbrace{ \frac{w}{b+w} \mal \frac{b}{b+w} \mal \frac{b}{w+b} + \dots}_{\ref{longer:winning}}\\
			&= \sum_{i \ge 1} \frac{b^i}{(s+w)^i}\mal \frac{w^{i-1}}{(b+w)^{i-1}}\\
								&= \sum_{i \ge 1} \frac{b^i w^{i-1}}{(b+w)^{i-1}}\\
								&= \sum_{i \ge 1} \frac{b}{b+w}\brackets{\frac{b^{i-1} w^{i-1}}{(b+w)^{2i -2}}}\\
								&= \sum_{i \ge 1} \frac{b}{b+w}\brackets{\frac{bw}{(b+w)^2}}^{i-1}\\
								&= \brackets{\frac{b}{b+w}} \mal \underbrace{\sum_{i \ge 1} \brackets{\frac{bw}{(b+w)^2}}^{i-1}}_{\text{geom. series conv. for } \abs{\frac{bw}{(b+w)^2}}\overset{(*)}{<}1}\\
								&= \frac{b}{b+w} \mal \frac{b^2 + 2bw + w^2}{b^2+sw+w^2}\\
								&= \frac{s\mal (b+w)}{s^2 + sw + w^2}
			\end{align*}
			$(*)$: holds here because $\abs{\frac{bw}{(b+w)^2}} < 1$, $b$ and $w$ are whole numbers!\\
			With which probability wins White?\\
			There is only a choice between black or white, so this can be easily calculated for white:
			\begin{align*}
				\Meas(w) &= 1 - \Meas(b)\\
				&= 1 - \frac{b(b+w)}{s^2 + bw + w^2}\\
				&= \frac{b^2 + bw + w^2 - b^2 - bw}{b^2 + bw + w^2}\\
				&= \frac{w^2}{b^2 + bw + w^2}
			\end{align*}
		\end{enumerate}
		\item For the same winning probability for $b \and w$, i found two solutions:
		\begin{enumerate}[label=\arabic*)]
			\item $\Meas(b) = \Meas(w)$:
			\begin{align*}
				\frac{b(b+w)}{b^2+bw+w^2} &= \frac{w^2}{b^2 + bw + w^2}\\
				b^2+bw) &= w^2 \quad \text{ divide by $w^2$, because $w^2 >0$}\\
				1 &= \frac{b^2}{w^2} + \frac{bw}{w^2}\\
				1 &= \brackets{\frac{b}{w}}^2 + \frac{b}{w} \quad \text{ set: } x:= \frac{b}{w}\\
				1 &= x^2 + x \implies x_{1,2} = \frac{-1 \pm \sqrt{5}}{2}
			\end{align*}
			Now we need to exclude one of the solutions. $x$ can not be $\frac{-1 - \sqrt{5}}{2}$, because $x$ has to be a positive number $\implies \frac{-1 + \sqrt{5}}{2}$ and this is also the golden ratio and i get for the winning ratio $b:w \approx 0,61$.
			\item 
			\begin{align*}
				\frac{b(b+w)}{b^2 + bw + w^2} 
				&= 1 - \frac{b(b+w)}{b^2 + bw + w^2}\\
				2 \mal \frac{b(b+w)}{b^2 + bw + w^2} 
				&= 1\\
				\frac{b(b+w)}{b^2 + bw + w^2} 
				&= \frac{1}{2} \quad \text{ set } \Psi := \frac{b}{b+w}\\
				\frac{\Psi}{\Psi^2 - \Psi + 1} 
				&= \frac{1}{2}\\
				\Psi^2 - 3\Psi + 1 
				&= 0\\
				\intertext{roots:}
				\Psi_{1,2} 
				&= \frac{3 \pm \sqrt{5}}{2}
			\end{align*}
				So Again can exclude $\frac{3 + \sqrt{5}}{2}$ as solution to the problem, because it would be $\Psi > 1$ and that is not possible $\implies \Psi = \frac{3 - \sqrt{5}}{2} \approx 0,38$, this means the ratio black versus $(b+w)$ is $\approx 0,38$ and ratio white versus $(b+w)$ is $\approx 0,38$ and this implies $\frac{b}{w} \approx \frac{0,38}{0,62} \approx 0,61$. 
			\end{enumerate}
			\item No idea. :( 
		\end{enumerate}
	\end{solution}

%%%%%%%%%%%%%%%%%%%% Aufgabe 4 %%%%%%%%%%%%%%%%%%%%%%%%%%%%%%%%%%%%%%%%%%%%%%%%%%