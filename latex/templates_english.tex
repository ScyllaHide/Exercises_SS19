% This work is licensed under the Creative Commons
% Attribution-NonCommercial-ShareAlike 4.0 International License. To view a copy
% of this license, visit http://creativecommons.org/licenses/by-nc-sa/4.0/ or
% send a letter to Creative Commons, PO Box 1866, Mountain View, CA 94042, USA.

% Du kannst dir jederzeit neue Templates definieren.
% Schreibe dir einfach einen eigenen Command in dieses File.

\newcommand{\TemplateUebung}[3]{
	% Argument no. 1 is the author
	% Argument no. 2 is the name of the lecture
	% Argument no. 3 is the worksheet number

	\documentclass{scrartcl} 
	
	\input{\directoryPrefix packages_english}
	\input{\directoryPrefix theoremenvironments_english}
	\input{\directoryPrefix commands}
	\input{\directoryPrefix commands_Willi}

	\author{#1}
	\parindent0cm %Ist wichtig, um führende Leerzeichen zu entfernen

	\usepackage{scrlayer-scrpage}
	\clearscrheadfoot
	
	\ihead{#1}
	\chead{}
	\ohead{#2}
	\ifoot{Sheet #3}
	\cfoot{Version: \today}
	\ofoot{Page \pagemark}
}

\newcommand{\TemplateSummary}[2]{
	% Argument no. 1 is the author
	% Argument no. 2 is the name of the lecture (short version, should match the file name of the corresponding local command file. See code below.)

	\documentclass[12pt]{scrartcl}
	\usepackage{palatino,setspace,fancyhdr}
	\usepackage[left=10mm,right=10mm,top=25mm,bottom=25mm]{geometry}
	\onehalfspacing
	\pagestyle{fancy}
	
	\chead{Summary #2}
	\lfoot{Version: \today}
	\rfoot{Page \thepage}
	\lhead{}
	\rhead{#1}

	\input{\directoryPrefix packages_english}
	\input{\directoryPrefix theoremenvironments_english}
	\input{\directoryPrefix commands}
	\input{\directoryPrefix commands_Willi}
	\IfFileExists{./commands_#2}{
		\input{commands_#2}
	}{}

	% Platzeinsparungen:
	\setitemize{leftmargin=*} % itemize-Umbegungen werden nicht eingerückt
	\setkomafont{section}{\large}
}
